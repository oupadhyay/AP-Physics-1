\documentclass[]{scrartcl}
\usepackage{amsmath}
\usepackage[utf8]{inputenc}
\usepackage[usenames]{color}
\usepackage{amsmath}
\usepackage{amssymb}
\usepackage{graphicx}
\usepackage{fancyhdr}
\usepackage{lmodern}
\usepackage{float}
\usepackage{MnSymbol,wasysym}
\renewcommand{\familydefault}{\sfdefault}

\title{Unit 5 - 7 Progress Checks}
\author{Ojasw Upadhyay}
\date{Period 5}

\begin{document}

\maketitle
\section{Unit 5}
\subsection*{Question 1}
\subsubsection*{Part A}
The conservation of energy will be used to determine the velocity of the block at the bottom of the ramp. In addition, the conservation of momentum will be used to solve for the velocity after the inelastic collision.
\begin{align}
M_XgH_X &= \dfrac12 M_Xv_X^2 \\
v_X &= \sqrt{2gH_X} \\
M_Xv_1 + 0 &= (M_X + M_Y)v_s \\
v_s &= \boxed{\dfrac{M_X\sqrt{2gH_X}}{M_X + M_Y}}
\end{align}

\subsubsection*{Part B}
\begin{enumerate}
\item Measure masses of block X and Y, but keep constant throughout the experiment.
\item Measure $H_X$ from the ground. This will be the independent variable of the experiment as we do trials with different heights.
\item Measuring the height from the ground at each run, slide the block from varying heights and use a motion probe to measure the velocity of the two-block system. The motion probe should be placed at one end of the table (either measuring velocity away or towards the probe).
\item Ensure that the velocity is measured only after the two blocks have collided and stuck together.
\item Repeat multiple times in trials to reduce experimental uncertainty.
\item The table can be recreated from the procedure above. :)
\end{enumerate}
\subsubsection*{Part C}
The predicted speeds are systematically smaller than the actual speed, motivating a force that always applies against motion. The force that follows this is friction. While the students assumed that the surface was frictionless, in reality, there was some friction between the blocks, which would cause it to lose energy to heat and therefore slows it down.
\subsection*{Question 2}
\subsubsection*{Part A}
The horizontal component has no change in momentum as the normal force only applies perpendicular to the surface which is horizontal; hence, there is no horizontal change in momentum. In the vertical component, there is a change in momentum upwards.
\subsubsection*{Part B}
In the real world, as the ball travels through the air, it meets air resistance or drag. This would act like friction, causing the system to lose usable energy ($E_a$). Moreover, when the ball bounces on the ground, as no collision is perfectly elastic, there is some energy transfered to the ground, which causes a further decrease in energy ($E_g$). This would mean that by the time the ball bounces, due to the conservation of energy,
$$E_{\text{total}} = E_0 - E_a - E_g$$
With this, the ball would have a smaller overall momentum and due to the conservation of momentum, the momentum of the ball of will be of the lower momentum. Finally, the ball would also lose usable energy as it travels through the air upwards due to drag. Due to these abundant friction forces, the height after the bounce will not be as high as when it started.
\section*{Unit 6}
\subsection*{Question 1}
\subsubsection*{Part A}
To determine the acceleration due to gravity, the students will be using a pendulum and its relevant equations:
$$T = 2\pi\sqrt{\dfrac{l}{g}}\Longrightarrow  \boxed{g = \dfrac{4\pi^2 l}{T^2}}$$
\subsubsection*{Part B}
\begin{enumerate}
  \item Take a string of a measured length using a meterstick.
  \item Suspend the string a pivot point above the ground.
  \item Raise the ball to a small angle to ensure simple harmonic motion.
  \item Measure the period of the motion using a stopwatch.
  \item To reduce experimental error, complete and log multiple trials.
  \item The table can be filled using the procedure above.
\end{enumerate}
\subsubsection*{Part C}
The vertical axis is the length of the string. The horizontal axis the period of the pendulum squared. As $\dfrac{g}{4\pi^2} = \dfrac{l}{T^2}$, we can find that the acceleration due to gravity, $g$, is equal to $4\pi^2 \times m$.
\subsubsection*{Part D}
As the entirety of the sphere's kinetic energy has been converted to potential energy, the iniital velocity of the sphere is 0, the sphere will travel straight down as the only force on the sphere is gravity. Moreover, its acceleration is equal to the acceleration due to gravity.

\end{document}

\documentclass[]{scrartcl}
\usepackage{amsmath}
\usepackage[utf8]{inputenc}
\usepackage[usenames]{color}
\usepackage{amsmath}
\usepackage{amssymb}
\usepackage{graphicx}
\usepackage{fancyhdr}
\usepackage{lmodern}
\usepackage{float}
\usepackage{MnSymbol,wasysym}
\renewcommand{\familydefault}{\sfdefault}

\title{Unit 5 - 7 Progress Checks}
\author{Ojasw Upadhyay}
\date{Period 5}

\begin{document}

\maketitle
\section*{Unit 5}
\subsection*{Question 1}
\subsubsection*{Part A}
  The conservation of energy will be used to determine the velocity of the block at the bottom of the ramp. In addition, the conservation of momentum will be used to solve for the velocity after the inelastic collision.
  \begin{align*}
    M_XgH_X &= \dfrac12 M_Xv_X^2 \\
    v_X &= \sqrt{2gH_X} \\
    M_Xv_1 + 0 &= (M_X + M_Y)v_s \\
    v_s &= \boxed{\dfrac{M_X\sqrt{2gH_X}}{M_X + M_Y}}
  \end{align*}
\subsubsection*{Part B}
  \begin{enumerate}
    \item Measure masses of block X and Y, but keep constant throughout the experiment.
    \item Measure $H_X$ from the ground. This will be the independent variable of the experiment as we do trials with different heights.
    \item Measuring the height from the ground at each run, slide the block from varying heights and use a motion probe to measure the velocity of the two-block system. The motion probe should be placed at one end of the table (either measuring velocity away or towards the probe).
    \item Ensure that the velocity is measured only after the two blocks have collided and stuck together.
    \item Repeat multiple times in trials to reduce experimental uncertainty.
    \item The table can be recreated from the procedure above. :)
  \end{enumerate}
\subsubsection*{Part C}
  The predicted speeds are systematically smaller than the actual speed, motivating a force that always applies against motion. The force that follows this is friction. While the students assumed that the surface was frictionless, in reality, there was some friction between the blocks, which would cause it to lose energy to heat and therefore slows it down.
\subsection*{Question 2}
\subsubsection*{Part A}
  The horizontal component has no change in momentum as the normal force only applies perpendicular to the surface which is horizontal; hence, there is no horizontal change in momentum. In the vertical component, there is a change in momentum upwards.
\subsubsection*{Part B}
  In the real world, as the ball travels through the air, it meets air resistance or drag. This would act like friction, causing the system to lose usable energy ($E_a$). Moreover, when the ball bounces on the ground, as no collision is perfectly elastic, there is some energy transfered to the ground, which causes a further decrease in energy ($E_g$). This would mean that by the time the ball bounces, due to the conservation of energy,
  $$E_{\textrm{total}} = E_0 - E_a - E_g$$
  With this, the ball would have a smaller overall momentum and due to the conservation of momentum, the momentum of the ball of will be of the lower momentum. Finally, the ball would also lose usable energy as it travels through the air upwards due to drag. Due to these abundant friction forces, the height after the bounce will not be as high as when it started.

\section*{Unit 6}
\subsection*{Question 1}
\subsubsection*{Part A}
  To determine the acceleration due to gravity, the students will be using a pendulum and its relevant equations:
  $$T = 2\pi\sqrt{\dfrac{l}{g}}\Longrightarrow  \boxed{g = \dfrac{4\pi^2 l}{T^2}}$$
\subsubsection*{Part B}
  \begin{enumerate}
    \item Take a string of a measured length using a meterstick.
    \item Suspend the string a pivot point above the ground.
    \item Raise the ball to a small angle to ensure simple harmonic motion.
    \item Measure the period of the motion using a stopwatch.
    \item To reduce experimental error, complete and log multiple trials.
    \item The table can be filled using the procedure above.
  \end{enumerate}
\subsubsection*{Part C}
  The vertical axis is the length of the string. The horizontal axis the period of the pendulum squared. As $\dfrac{g}{4\pi^2} = \dfrac{l}{T^2}$, we can find that the acceleration due to gravity, $g$, is equal to $4\pi^2 \times m$.
\subsubsection*{Part D}
  As the entirety of the sphere's kinetic energy has been converted to potential energy, the iniital velocity of the sphere is 0, the sphere will travel straight down as the only force on the sphere is gravity. Moreover, its acceleration is equal to the acceleration due to gravity.
\subsection*{Question 2}
\subsubsection*{Part A}
  The force is a negatively sloped line that travels through the origin as when the block moves away (+) from the equilibrium point will feel a force towards the equilibrium position (-). Moreover, when the block is at the equilibrium position, there is no force, so the line must travel through the origin.
\subsubsection*{Part B}
  At the position $x = 0.25$ meters, we see that the potential energy of the spring is 10 J. Using the equation for the potential energy of a spring, we get:
  \begin{align*}
    \dfrac12 kx^2 &= 10\;\mathrm{J} \\
    k &= \dfrac{20}{x^2} = \dfrac{20}{0.25^2} = 320 \;\mathrm{J/m^2} \\
    T &= 2\pi\sqrt{\dfrac{m}{k}} = 2\pi\sqrt{\dfrac{0.4}{320}} = \boxed{0.222\;\mathrm{s}}
  \end{align*}
\subsubsection*{Part C}
  Because of the conservation of energy, the  total kinetic energy and potential energy at any horizontal position must be the same. Hence, if $p(x)$ describes the potential energy graph and the maximum potential energy is 10 J, then the kinetic energy fucntion would be $k(x) = 10 - p(x)$. This would look if the potential energy graph was flipped vertically.

\section*{Unit 7}
\subsection*{Question 1}
\subsubsection*{Part A}
  Assuming that A and B are equidistat from the center of mass, then the least tension(1) is in Location C as it furthest away from the center of mass. Then, both Location A and B will be ranked 2 as they require more tension than Location C.
\subsubsection*{Part B}
  Angular acceleration is created by a torque, which is the perpendicular component of the force. At $t_1$, the gravitational force is almost entirely perpendicular to the sign, hence the torque ($\tau = rF\sin\theta$) is larger at $t_1$ than $t_2$ or $t_3$. Since $\tau_1 > \tau_2 > \tau_3$,
  \begin{align*}
    \alpha &= \dfrac{\vec{\tau}}{I} &\\
    \dfrac{\tau_1}{I} &> \dfrac{\tau_2}{I} > \dfrac{\tau_3}{I}
  \end{align*}
  These equations show how the decrease in the torque from gravity leads to a smaller torque, which leads to a smaller angular acceleration. Hence, the ranking is $\boxed{t_1 > t_2 > t_3}$. \\ \\
  To find the equation for the angular acceleration,
  \begin{align*}
    \alpha &= \dfrac{\vec{\tau}_{\textrm{net}}}{I} \\
     &= \dfrac{\dfrac12 L_s \cdot M_s g\sin\theta}{\dfrac13 M_sL_s^2} \\
     &= \boxed{\dfrac{3g\sin\theta}{2L_s}} \\
  \end{align*}
\subsubsection*{Part C}
  First of all, the input to the $\sin$ should be defined in terms of $t$ as without it, the formula seems to use the same angle throughout the motion of the sign. However, physically, it makes sense that an increase in gravitational acceleration would increase the angular velocity and a decrease in the length of the sign would decrease the period of the string, which would increase the angular velocity. Hence, beyond the requirement for $\theta$ being representative of a value that changes with time, the function makes sense physically. Moreover, the function is derived from $\omega = \omega_0 + \alpha t$.
\subsubsection*{Part D}
  As $\theta \rightarrow 90$, the $omega$  also goes to 0. Hence the best trignometric function for the data collected would be $\cos\theta$ as $\cos 90 = 0$. Moreover, as stated in part c, the gravitational acceleration of the block is proportional to the angular velocity as an increase in the acceleration would increase the speed of the block. Hence, the first equatoin best matches the data collected.
\subsection*{Question 2}
\subsubsection*{Part A}
  There will be the force of gravity that goes straight down from the center of mass of the hoop. Also, there will be a normal force whose magnitude is equal to the perpendicular component of gravity and a frictional force, which is a fraction of the normal force (<1) That points from the contact point of the hoop to the ground and is pointed in opposition to motion.
\subsubsection*{Part B}
  \begin{align*}
    KE_{\textrm{tot}} &= \dfrac12 m\omega^2r^2 + \dfrac12 I\omega^2 \tag{equation} \\
    KE_{\textrm{tot}} &= \omega_0^2\left(\dfrac12 m_0 r_0^2 + \dfrac12 I_h\right) \tag{hoop} \\
    \omega_0 &= \sqrt{\dfrac{2KE_{\textrm{tot}}}{m_0 r_0^2 + I_h}} \\ \\
    KE_{\textrm{tot}} &= \omega_1^2\left(\dfrac12 m_0 r_0^2 + \dfrac12 I_d\right) \tag{solid disk} \\
    \omega_1 &= \sqrt{\dfrac{2KE_{\textrm{tot}}}{m_0 r_0^2 + I_d}} \\ \\
    0 &= \dfrac12 gt^2 - H_0 \tag{finding time} \\
    t &= \dfrac{\sqrt{2gH_0}}{g} \\
    \Delta x &= r\omega t \tag{dependent on $\omega$}
  \end{align*}
  Now, we can clearly see that the angular velocity of the hoop will be greater than that of a disk as $I_h < I_d$. Then, when calculating the total distance traveled by the disk or the hoop, we can see that since the time to fall is the same, the distance only depends on the angular velocity. In conclusion, since the hoop has a smaller rotational inertia, the distance it travels is larger that that of the solid disk.

\end{document}
